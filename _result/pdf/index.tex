% -*- mode: latex; -*- mustache tags:  
\documentclass[10pt,twoside,english]{_support/latex/sbabook/sbabook}
\let\wholebook=\relax

\usepackage{import}
\subimport{_support/latex/}{common.tex}

%=================================================================
% Debug packages for page layout and overfull lines
% Remove the showtrims document option before printing
\ifshowtrims
  \usepackage{showframe}
  \usepackage[color=magenta,width=5mm]{_support/latex/overcolored}
\fi


% =================================================================
\title{Pharo Graphics}
\author{The Pillar team}
\series{Square Bracket tutorials}

\hypersetup{
  pdftitle = {Pharo Graphics},
  pdfauthor = {The Pillar team},
  pdfkeywords = {graphics, canvas}
}


% =================================================================
\begin{document}

% Title page and colophon on verso
\maketitle
\pagestyle{titlingpage}
\thispagestyle{titlingpage} % \pagestyle does not work on the first one…

\cleartoverso
{\small

  Copyright 2017 by The Pillar team.

  The contents of this book are protected under the Creative Commons
  Attribution-ShareAlike 3.0 Unported license.

  You are \textbf{free}:
  \begin{itemize}
  \item to \textbf{Share}: to copy, distribute and transmit the work,
  \item to \textbf{Remix}: to adapt the work,
  \end{itemize}

  Under the following conditions:
  \begin{description}
  \item[Attribution.] You must attribute the work in the manner specified by the
    author or licensor (but not in any way that suggests that they endorse you
    or your use of the work).
  \item[Share Alike.] If you alter, transform, or build upon this work, you may
    distribute the resulting work only under the same, similar or a compatible
    license.
  \end{description}

  For any reuse or distribution, you must make clear to others the
  license terms of this work. The best way to do this is with a link to
  this web page: \\
  \url{http://creativecommons.org/licenses/by-sa/3.0/}

  Any of the above conditions can be waived if you get permission from
  the copyright holder. Nothing in this license impairs or restricts the
  author's moral rights.

  \begin{center}
    \includegraphics[width=0.2\textwidth]{_support/latex/sbabook/CreativeCommons-BY-SA.pdf}
  \end{center}

  Your fair dealing and other rights are in no way affected by the
  above. This is a human-readable summary of the Legal Code (the full
  license): \\
  \url{http://creativecommons.org/licenses/by-sa/3.0/legalcode}

  \vfill

  % Publication info would go here (publisher, ISBN, cover design…)
  Layout and typography based on the \textcode{sbabook} \LaTeX{} class by Damien
  Pollet.
}


\frontmatter
\pagestyle{plain}

\tableofcontents*
\clearpage\listoffigures

\mainmatter


\chapter{Vector graphics in Athens}
There are two different computer graphics: vector and raster graphics.
Raster graphics represents images as a collection of pixels. Vector graphics
is the use of geometric primitives such as points, lines, curves, or polygons
to represent images. These primitives are created using mathematical equations.

Both types of computer graphics have advantages and disadvantages.
The advantages of vector graphics over raster are:

\begin{itemize}
    \item smaller size,
    \item ability to zoom indefinitely,
    \item moving, scaling, filling, and rotating do not degrade the quality of an image.
\end{itemize}

Ultimately, picture on a computer are displayed on a screen with a specific
display dimension. However, while raster graphic doesn't scale very well when
the resolution differ too much from the picture resolution, vector graphics
are rasterized to fit the display they will appear on. Rasterization is the
technique of taking an image described in a vector graphics format and
transform it into a set of pixels for output on a screen.

Morphic is the way to do graphics with Pharo.
However, most existing canvas are pixel based, and not vector based.
This can be an issue with current IT ecosystems, where the resolution can differ from machine to machine (desktop, tablet, phones, etc...)

Enter Athens, a vector based graphic API. Under the scene, it can either use
balloon Canvas, or the cairo graphic library for the rasterization phase.

\section{Example}
Here it would be nice to have an example

\section{Athens details}
\textcode{AthensSurface} and its subclass \textcode{AthensCairoSurface} instances represent a surface.
A surface is the area in pixel where your drawing will be rendered.
You  never draw directly on the surface.
Instead, you specify what you want to display on the canvas, and Athens will render it on the area specified by the surface.

The class \textcode{AthensCanvas} is the central object used to perform drawing on an \textcode{AthensSurface}
A canvas is not directly instanciated but used through a message such as:\textcode{surface drawDuring: {[}:canvas \textbar{} .... {]}}

The Athens drawing model relies on a three layer model. Any drawing process
takes place in three steps:

\begin{itemize}
    \item First a \textcode{path} is created, which includes one or more vector primitives , i.e., circles, lines, TrueType fonts, B\'{e}zier curves, etc...
    \item Then painting must be defined, which may be a color, a color gradient, a bitmap or some vector graphics
    \item Finally the result is drawn to the Athens surface, which is provided by the back-end for the output.
\end{itemize}

\section{Path}
Athens has always an active path.
Use \textcode{AthensPathBuilder} or \textcode{AthensSimplePathBuilder} to build a path.
They will assemble path segments for you.

The method \textcode{createPath:} exists in all important Athens class: \textcode{AthensCanvas},\textcode{AthensSurface}, and  \textcode{AthensPathBuilder}.
The message \textcode{createPath: aPath}

Using it returns a new path:
\begin{displaycode}{smalltalk}
surface createPath: [ :builder |
 builder
  absolute;
  moveTo: 100@100;
  lineTo: 100@300;
  lineTo: 300@300;
  lineTo: 300@100;
  close ].
\end{displaycode}

Here are some helper messages in \textcode{AthensSimplePathBuilder}:

\begin{itemize}
    \item \textcode{pathStart}
    \item \textcode{pathBounds} gives the limit of the bounds associated to the path
\end{itemize}

If you want to build path using only straight line, you can use the class \textcode{AthensPolygon}.

\begin{fullwidthtabular}{lll}
\toprule
path builder Messages & Object Segment & comment \\
\bottomrule
\end{fullwidthtabular}

-

\begin{fullwidthtabular}{lll}
\toprule
ccwArcTo: angle: & AthensCCWArcSegment & counter clock wise segment \\
cwArcTo:angle: & AthensCWArcSegment & clock wise segment \\
lineTo: & AthensLineSegment & straight line \\
moveTo: & AthensMoveSegment & start a new contour \\
curveVia: to: & AthensQuadSegment & quadric bezier curve \\
curveVia: and: to: & AthensCubicSegment & Cubic bezier curve \\
reflectedCurveVia: to: & AthensCubicSegment & Reflected cubic bezier curve \\
string: font: &  & specific to cairo \\
close & AthensCloseSegment & close the current contour \\
\bottomrule
\end{fullwidthtabular}

\section{Coordinate class: \textbf{Absolute} or \textbf{Relative}}
absolute: absolute coordinate from surface coordinate.
This will draw a square in a surface which extent is 400@400 using absolute move.
\begin{displaycode}{smalltalk}
builder absolute;
 moveTo: 100@100;
 lineTo: 100@300;
 lineTo: 300@300;
 lineTo: 300@100;
 close
\end{displaycode}

relative: each new move is relative to the previous one.
This will draw a square in a surface which extent is 400@400 using relative move.
\begin{displaycode}{smalltalk}
 builder relative ;
  moveTo: 100@100;
  lineTo: 200@0;
  lineTo: 0@200;
  lineTo: -200@0;
  close
\end{displaycode}

cwArcTo:angle: and ccwArcTo: angle: will draw circular arc, connecting
previous segment endpoint and current endpoint of given angle, passing in
clockwise or counter clockwise direction. The angle must be specified in Radian.

\subsection{curveVia: to: and \textbar{}curveVia: and: to:}
This call is related to bezier curve. A B\'{e}zier curve consists of two or more
control points, which define the size and shape of the line. The first and
last points mark the beginning and end of the path, while the intermediate
points define the path's curvature.

More detail on Bezier curve on available at: https://pomax.github.io/bezierinfo/

\subsection{path transformation.}
A path can be rotated, translated and scaled so you can adapt it to your need.
For example, you can define a path in your own coordinate system, and then
scale it to match the size of your surface extent.

\section{The different type of painting.}
Paints can be created either from the surface or directly from a class that will
do the call to the surface for you.

any object can be treated as paint:
- \textcode{athensFillPath: aPath on: aCanvas}
- \textcode{athensFillRectangle: aRectangle on: aCanvas}
- \textcode{asStrokePaint}

\begin{fullwidthtabular}{ll}
\toprule
surface message & comment \\
\sout{}\sout{\textasciitilde{}\textasciitilde{}}\sout{\textasciitilde{}\textasciitilde{}\textasciitilde{}\textasciitilde{}}\sout{\textasciitilde{}\textasciitilde{}\textasciitilde{}\textasciitilde{}\textasciitilde{}\textasciitilde{}}\sout{\textasciitilde{}\textasciitilde{}\textasciitilde{}\textasciitilde{}\textasciitilde{}\textasciitilde{}\textasciitilde{}\textasciitilde{}}\sout{\textasciitilde{}\textasciitilde{}\textasciitilde{}\textasciitilde{}\textasciitilde{}\textasciitilde{}\textasciitilde{}\textasciitilde{}\textasciitilde{}\textasciitilde{}}\sout{\textasciitilde{}\textasciitilde{}\textasciitilde{}\textasciitilde{}\textasciitilde{}\textasciitilde{}\textasciitilde{}\textasciitilde{}\textasciitilde{}\textasciitilde{}\textasciitilde{}\textasciitilde{}}\sout{\textasciitilde{}\textasciitilde{}\textasciitilde{}\textasciitilde{}\textasciitilde{}\textasciitilde{}\textasciitilde{}\textasciitilde{}\textasciitilde{}\textasciitilde{}\textasciitilde{}\textasciitilde{}\textasciitilde{}\textasciitilde{}}\sout{\textasciitilde{}\textasciitilde{}\textasciitilde{}\textasciitilde{}\textasciitilde{}\textasciitilde{}\textasciitilde{}\textasciitilde{}\textasciitilde{}\textasciitilde{}\textasciitilde{}\textasciitilde{}\textasciitilde{}\textasciitilde{}\textasciitilde{}\textasciitilde{}}\sout{\textasciitilde{}\textasciitilde{}\textasciitilde{}\textasciitilde{}\textasciitilde{}\textasciitilde{}\textasciitilde{}\textasciitilde{}\textasciitilde{}\textasciitilde{}\textasciitilde{}\textasciitilde{}\textasciitilde{}\textasciitilde{}\textasciitilde{}\textasciitilde{}\textasciitilde{}\textasciitilde{}}\sout{\textasciitilde{}\textasciitilde{}\textasciitilde{}\textasciitilde{}\textasciitilde{}\textasciitilde{}\textasciitilde{}\textasciitilde{}\textasciitilde{}\textasciitilde{}\textasciitilde{}\textasciitilde{}\textasciitilde{}\textasciitilde{}\textasciitilde{}\textasciitilde{}\textasciitilde{}\textasciitilde{}\textasciitilde{}\textasciitilde{}}\sout{\textasciitilde{}\textasciitilde{}\textasciitilde{}\textasciitilde{}\textasciitilde{}\textasciitilde{}\textasciitilde{}\textasciitilde{}\textasciitilde{}\textasciitilde{}\textasciitilde{}\textasciitilde{}\textasciitilde{}\textasciitilde{}\textasciitilde{}\textasciitilde{}\textasciitilde{}\textasciitilde{}\textasciitilde{}\textasciitilde{}\textasciitilde{}\textasciitilde{}}- & \sout{}\sout{\textasciitilde{}\textasciitilde{}}\sout{\textasciitilde{}\textasciitilde{}\textasciitilde{}\textasciitilde{}}\sout{\textasciitilde{}\textasciitilde{}\textasciitilde{}\textasciitilde{}\textasciitilde{}\textasciitilde{}}\sout{\textasciitilde{}\textasciitilde{}\textasciitilde{}\textasciitilde{}\textasciitilde{}\textasciitilde{}\textasciitilde{}\textasciitilde{}}\sout{\textasciitilde{}\textasciitilde{}\textasciitilde{}\textasciitilde{}\textasciitilde{}\textasciitilde{}\textasciitilde{}\textasciitilde{}\textasciitilde{}\textasciitilde{}} \\
createFormPaint: & create paint from a Form \\
createLinearGradient: start: stop: & linear gradient paint \\
createRadialGradient: center: radius: & Radial gradient paint \\
createRadialGradient: center: radius: focalPoint: & Radial gradient paint \\
createShadowPaint: & ??? \\
createSolidColorPaint: & fill paint \\
createStrokePaintFor: & stroke paint \\
\bottomrule
\end{fullwidthtabular}

a Canvas define its paint method we will see in detail below:

\begin{itemize}
    \item setPaint:
    \item setStrokePaint:
\end{itemize}

\section{Stroke paint (a pen that goes around the path)}
The \textcode{createStrokePaintFor:} operation takes a virtual pen along the path. It allows the source
to transfer through the mask in a thin (or thick) line around the path

\textcode{AthensStrokePaint} represents a stroke paint.

\section{Solid paint (a pen that fill the path)}
The \textcode{createSolidColorPaint} operation instead uses the path like the lines of a coloring book,
and allows the source through the mask within the hole whose boundaries are the
path. For complex paths (paths with multiple closed sub-paths—like a donut—or
paths that self-intersect) this is influenced by the fill rule

SD: We need examples

\section{Gradient}
Gradient will let you create gradient of color, either linear, or radial.

The color ramp is a collection of associations with keys - floating point values
between 0 and 1 and values with Colors, for example:



\bibliographystyle{alpha}
\bibliography{book.bib}

% lulu requires an empty page at the end. That's why I'm using
% \backmatter here.
\backmatter

% Index would go here

\end{document}
